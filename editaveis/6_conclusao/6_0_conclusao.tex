\chapter[Conclusão]{Conclusão}


O objetivo deste trabalho é validar a possibilidade de desenvolver uma aplicação descentralizada, por meio da tecnologia blockchain, que garantisse a integridade dos dados relacionados ao registro de infrações de trânsito e outros casos previstos no Código de Trânsito Brasileiro assim como demonstrar a viabilidade da utilização de uma aplicação descentralizada em um contexto hoje totalmente centralizado.

A solução desenvolvida foi implementada utilizando-se de ferramentas e tecnologias atuais, além de seguir padrões de código e arquiteturais amplamente conhecidos pela comunidade de software, podendo ser assim ser facilmente escalada e aprimorada ao longo do tempo para a utilização em outros contextos.

A utilização de contratos inteligentes para o registro de infrações de trânsito traz fatores positivos a este contexto, como por exemplo uma maior auditabilidade das informações registradas assim como uma integridade dos dados dentro do sistema, trazendo uma maior confiabilidade da população em frente aos órgãos responsáveis pelo trânsito no Brasil. Porém a utilização do contrato utilizando esta tecnologia neste contexto também pode esbarrar em um problema caso ocorra a modificação da constituição que hoje rege esse sistema pois os contratos são imutáveis e não podem ser alterados após a sua criação, fazendo assim com que uma mudança mesmo que simples possa levar a necessidade da criação de um novo contrato e assim todas as transações e dados armazenados no contrato anterior precisem ser migrados ou uma camada intermediária seja criada para adaptar esse novo contexto.

Outro fator de atenção na utilização de uma rede descentralizada neste contexto é o tempo para a confirmação de uma transação solicitada, como por exemplo o registro de uma infração. Este tempo é determinado pelo tamanho e tipo da rede blockchain utilizada, sendo assim importante analisar o contexto e avaliar o que melhor se aplica podendo este problema ser mitigado pela utilização de uma rede robusta ou por meio de uma melhor experiência de usuário dentro da aplicação cliente caso seja necessário um intervalo considerado grande até a confirmação de uma transação.

A solução desenvolvida validou o objetivo proposto demonstrando que apesar de possíveis problemas, que são inerentes a todos os tipos de aplicações, é possível a utilização da tecnologia blockchain em um contexto de registro de infrações de trânsito que na data de publicação deste trabalho é considerado burocrático, ineficiente e centralizado sem que hajam prejuízos significativos a seus utilizadores. 

A solução pode ser abstraída para outros contextos da esfera pública de administração, onde a criação de uma rede blockchain governamental possibilitaria que diversos serviços fossem disponibilizados de forma descentralizada facilitando o acesso e a utilização do mesmo por meio da população. E por meio de incentivos governamentais essa rede poderia ser expandida para diversos nós trazendo mais integridade, confiabilidade e auditabilidade aos dados de diversos órgãos.