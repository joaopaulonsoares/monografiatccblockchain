\chapter[Introdução]{Introdução}
%\addcontentsline{toc}{chapter}{Introdução}




Não resta dúvida de que atualmente o trânsito seria inviável sem a presença de normas, de regras que regulam a movimentação e a ocupação do espaço viário. Toda a supervisão do trânsito (fiscalização, controle) está baseada num conjunto de leis e dispositivos, que constam nos códigos e regulamentos de cada país. Em toda a experiência mundial de trânsito, as normas juntamente com a fiscalização do seu cumprimento têm exercido um papel fundamental, na medida em que é muito notável a relação direta entre fiscalização-punição e comportamento adequado (não infração às normas) com a diminuição de acidentes, e inversamente entre a impunidade e o comportamento inadequado (infração às normas) com aumento de acidentes. Um sistema normativo é condição indispensável para a fluidez e segurança no trânsito, sendo esta, sempre relacionada (e medida) aos índices de acidentes \cite{article_transito_alcool}.

Em 2018 foram registradas pelas autoridades de trânsito somente no Distrito Federal um total 2.740.685 infrações de trânsito, o que corresponde a uma média diária de aproximadamente 7500 infrações. Este número se torna ainda maior quando considerado os registros realizados por órgãos de outras unidades federativas, e também pelos órgãos responsáveis por fiscalizações em âmbito federal como a Polícia Rodoviária Federal.

A forma e o sistema em que o registro da infração de trânsito é feito depende de fatores como o órgão autuador e a localização geográfica da mesma. Esse forma de registro não padronizada, centralizada pelas autoridades e condicionado a diversos fatores fazem com que os serviços associados aos órgãos de trânsito, como aplicação de multas ou realização de possíveis recursos, se tornem ineficientes e lentos, o que eleva o custo de manutenção para o Estado e aumenta a insatisfação de todos os cidadãos que precisam utilizar algum destes serviços.

De forma a integrar esta capilaridade de sistemas presentes nas Unidades Federativas, o sistema RENAINF (Registro Nacional de Infrações de Trânsito) foi criado e é atualmente coordenado pelo DENATRAN (Departamento Nacional de Trânsito). Porém este sistema somente é utilizado para registrar as infrações de trânsito cometidas em unidade federada diferente daquela de onde o veículo estiver registrado e licenciado, bem como para o registro das infrações impostas pelas autoridades de trânsito federadas independente da vinculação de registro do veículo. Este sistema possibilita que o órgão autuador obtenha os dados necessários para registrar a informação da infração cometida e vincular estes débitos no Departamento Estadual de Trânsito  (DETRAN) de registro do veículo, não sendo assim um sistema realmente unificado para o registro de infrações de trânsito no território brasileiro \cite{renainf_fazenda_sp}.

A construção de um sistema nacional único e centralizado por uma autoridade como o DENATRAN poderia mitigar problemas decorrentes dessa capilaridade de sistemas, porém esta centralização acaba tendo seus pontos negativos como um maior custo de manutenção e maior dificuldade para garantir a integridade do sistema. Em meados de 2009 uma nova tecnologia denominada Blockchain foi proposta para a aplicação em uma aplicação monetária que não necessitava de uma unidade central de confiança para seu funcionamento, e logo observou-se que seu conceito poderia ser aplicado em contextos diferentes do que foi idealizado por possibilitar algumas vantagens como a eliminação de intermediários e o aumento da segurança com custo baixo \cite{beginnig_blockchain_bikramaditya}.


\section{Justificativa}

Blockchain possui um grande potencial para ser inserido diversos setores uma alguma forma ou de outra; e esta revolução já começou em algumas áreas, afetando enormemente o mercado de serviços financeiros. É difícil nomear um banco global ou uma entidade financeira que não explore o blockchain. Além do mercado financeiro, iniciativas já estão sendo tomadas em áreas como mídia e entretenimento, comercialização de energia, mercados de previsão, redes de varejo, sistemas de fidelidade, seguro, logística e cadeias de suprimento, registros médicos e também aplicações governamentais e militares \cite{beginnig_blockchain_bikramaditya}.

Considerando o potencial para aplicação desta tecnologia disruptiva, principalmente para aplicação em serviços considerados de ordem pública como os oferecidos e mantidos pelo Governo, e a existência de problemáticas como a elencada anteriormente em relação ao registro de infrações de trânsito. Este trabalho tem como proposta demonstrar a viabilidade da construção e utilização de uma aplicação descentralizada em um contexto hoje centralizado.

\section{Objetivos}

    \subsection{Objetivo Geral}
    
    O objetivo principal da pesquisa é desenvolver uma aplicação descentralizada para o registro de infrações de trânsito, assim como possibilitar por meio da ferramenta a realização de outros serviços previstos pelo CTB.
    
    \subsection{Objetivos Específicos}
    
    \begin{enumerate}
        \item Desenvolver uma aplicação descentralizada acessível a todos que desejarem;
        \item Estudar sobre a nova tecnologia \textit{blockchain} e como ela pode ser usada em diversas áreas;
        \item Demonstrar a viabilidade do uso de aplicações descentralizadas em um contexto atualmente centralizado;
    \end{enumerate}
    
    
\section{Estrutura do Documento}

Este documento está composto pelos seguintes capítulos:

\begin{enumerate}
    \item \textbf{Introdução:} A introdução contém um breve contexto e motivações para a realização deste trabalho;
    \item \textbf{Fundamentação Teórica:} A fundamentação teórica contém conceitos teóricos considerados necessários para uma melhor compreensão do trabalho que será desenvolvido;
    \item \textbf{Proposta de Trabalho:} Contém uma explanação da proposta de trabalho à ser desenvolvido;
    \item \textbf{Metodologia:} Contém descrição da forma que o projeto será desenvolvido, assim como metodologias, políticas e ferramentas que serão utilizadas.
\end{enumerate}
