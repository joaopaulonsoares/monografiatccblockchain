\section{Código de Trânsito Brasileiro}


    O Código de Trânsito Brasileiro(CTB) é um documento legal que foi instituído em 1997 pelo presidente da República por meio da Lei n 9.503, substituindo o Código Nacional de Trânsito de 1966 que era responsável por regular as normas de trânsito no Brasil até então. As regras de trânsito definidas no CTB são atualizadas todos os anos pelos poderes Executivo e Legislativo, por meio da publicação de leis que atualizam os artigos do CTB.

    O CTB assegura que o trânsito é um direito de todos e dever dos órgãos e entidades componentes do Sistema Nacional De Trânsito(SNT) adotar medidas destinadas a assegurar esse direito. Considera-se trânsito: "a utilização das vias por pessoas, veículos e animais, isolados ou em grupos, conduzidos ou não, para fins de circulação, parada, estacionamento e operação de carga ou descarga" \cite{codigo_transito_bra}.
    
    O Sistema Nacional de Trânsito é definido no CTB pelo artigo 5 como: "{...} conjunto de órgãos e entidades da União, dos Estados, do Distrito Federal e dos Municípios que tem por finalidade o exercício das atividades de planejamento, administração, normatização, pesquisa, registro e licenciamento de veículos, formação, habilitação e reciclagem de condutores, educação, engenharia, operação do sistema viário, policiamento, fiscalização, julgamento de infrações e de recursos e aplicação de penalidades".

    Segundo o artigo 7 do CTB, o Sistema Nacional de Trânsito é composto dos seguintes órgãos e entidades:
    
        \begin{quote}
            ``Art. 7º Compõem o Sistema Nacional de Trânsito os seguintes órgãos e entidades:

            \renewcommand{\theenumi}{\Roman{enumi}}%
                \begin{enumerate}
                    \item o Conselho Nacional de Trânsito - CONTRAN, coordenador do Sistema e órgão máximo normativo e consultivo;
                    \item os Conselhos Estaduais de Trânsito - CETRAN e o Conselho de Trânsito do Distrito Federal - CONTRANDIFE, órgãos normativos, consultivos e coordenadores;;
                    \item os órgãos e entidades executivos de trânsito da União, dos Estados, do Distrito Federal e dos Municípios;
                    \item os órgãos e entidades executivos rodoviários da União, dos Estados, do Distrito Federal e dos Municípios;
                    \item a Polícia Rodoviária Federal;
                    \item as Polícias Militares dos Estados e do Distrito Federal; e
                    \item as Juntas Administrativas de Recursos de Infrações - JARI.''
                 \end{enumerate}
         \end{quote}


\subsection{Infração de Trânsito}

    O CTB define(Cap.15,Art.161) infração como: 
    
         \begin{quote}
            ``{...} a inobservância de qualquer preceito deste Código, da legislação complementar ou das resoluções do CONTRAN, sendo o infrator sujeito às penalidades e medidas administrativas indicadas em cada artigo, além das punições previstas no Capítulo XIX.''
         \end{quote}
         
    A multa é uma das penalidades aplicadas aos motoristas que cometeram uma ou mais das infrações elencadas no CTB, onde a multa é uma penalidade relacionada a todos os outros tipos de infração. A lista das penalidades aplicadas está elencada no artigo 256, sendo estas:
    
    \begin{quote}
            \renewcommand{\theenumi}{\Roman{enumi}}%
            \begin{enumerate}
                \item advertência por escrito;
                \item \textbf{multa};
                \item suspensão do direito de dirigir;
                \item apreensão do veículo; (Revogado pela Lei nº 13.281, de 2016)
                \item cassação da Carteira Nacional de Habilitação;
                \item cassação da Permissão para Dirigir;
                \item frequência obrigatória em curso de reciclagem.''
             \end{enumerate}
    \end{quote}
    

    \subsubsection{Tipos de Infração}

    O CTB categoriza os tipos de infrações de trânsito em quatro diferentes categorias: leves, médias, graves, gravíssimas. Cada categoria possui um valor diferente de penalização, onde as infrações de natureza gravíssima podem ter seu custo aumentado pelos fatores multiplicadores previstos no parágrafo 2 do artigo 258.
    
    Os valores definidos no artigo 258 para cada natureza de infração são apresentados na Tabela \ref{tabela_precos_infracoes_categorias}:
    
    \begin{table}[H]
        \caption{Valor da penalização de cada categoria das infrações previstas no artigo 258 do CTB em 2019}
        \begin{tabular}{|l|c|ll}
        \cline{1-2}
        \multicolumn{1}{|c|}{\textbf{Categoria da Infração}} & \textbf{Valor} &  &  \\ \cline{1-2}
        Leve & R\$ 88,38 &  &  \\ \cline{1-2}
        Média & R\$ 130,16 &  &  \\ \cline{1-2}
        Grave & R\$ 195,23 &  &  \\ \cline{1-2}
        Gravíssima & R\$ 293,47 &  &  \\ \cline{1-2}
        \end{tabular}
        \label{tabela_precos_infracoes_categorias}
    \end{table}
    
    Além da penalização monetária aplicada, uma pontuação também é computada à CNH do condutor identificado como condutor no ato da infração de acordo com a natureza da mesma. A pontuação atribuída ao infrator é regulamentada pelo artigo 259 do CTB, que apresenta a seguinte redação:
        
        \renewcommand{\theenumi}{\Roman{enumi}}%
        \begin{quote}
            ``Art. 259. A cada infração cometida são computados os seguintes números de pontos:
            
                \begin{enumerate}
                  \item gravíssima - sete pontos;
                  \item grave - cinco pontos;
                  \item média - quatro pontos;
                  \item leve - três pontos.''
                \end{enumerate}
        \end{quote}
        
    \subsubsection{Auto de Infração}
    \label{estrutura_auto_infracao}

    Constatada a infração, será lavrado o Auto de Infração, que deverá conter os requisitos mínimos definidos pelo artigo 280 do CTB \cite{codigo_transito_bra}:

    \begin{quote}
        `` Art. 280. Ocorrendo infração prevista na legislação de trânsito, lavrar-se-á auto de infração, do qual constará:

            \renewcommand{\theenumi}{\Roman{enumi}}%
            \begin{enumerate}
              \item tipificação da infração;
              \item local, data e hora do cometimento da infração;
              \item caracteres da placa de identificação do veículo, sua marca e espécie, e outros elementos julgados necessários à sua identificação;
              \item o prontuário do condutor, sempre que possível;
              \item identificação do órgão ou entidade e da autoridade ou agente autuador ou equipamento que comprovar a infração;
              \item assinatura do infrator, sempre que possível, valendo esta como notificação do cometimento da infração.''
            \end{enumerate}
    \end{quote}



\subsection{Registro de Infrações de Trânsito}
\label{section_renainf}


    O registro das infrações de trânsito é feito pelo órgão ou entidade responsável pela jurisdição onde a infração foi cometida, sendo a mesma podendo ser salva em um sistema local do órgão caso a infração seja registrada na mesma unidade federativa que o veículo está licenciado ou no Registro Nacional de Infrações de Trânsito(RENAINF) caso o veículo não seja licenciado no estado onde ocorreu a infração. Por exemplo, no Distrito Federal os órgãos e entidades responsáveis por fazer fiscalizações e registrar infrações são: Departamento de Trânsito Do Distrito Federal (Detran-DF), Departamento de Estradas de Rodagem (DER) e Polícia Militar(PMDF).
    
    O RENAINF - Registro Nacional de Infrações de Trânsito é um sistema coordenado pelo Departamento Nacional de Trânsito(DENATRAN) que registra as infrações à legislação de trânsito cometidas em unidade federada diversa daquela onde o veículo estiver registrado e licenciado, bem como permite o registro das infrações impostas pela Polícia Rodoviária Federal(PRF), Agência Nacional de Transportes Terrestres(ANTT) e o Departamento Nacional de Infraestrutura de Transportes(DNIT), independente da vinculação de registro do veículo. Esse sistema possibilita que o órgão autuador tenha os dados necessários para notificar o proprietário do veículo sobre a infração cometida e sobre a respectiva penalidade, e vincular os débitos existentes no DETRAN de registro do veículo.\cite{renainf_fazenda_sp}
