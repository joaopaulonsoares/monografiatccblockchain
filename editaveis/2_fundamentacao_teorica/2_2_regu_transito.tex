\section{Código de Trânsito}

\subsection{Código de Trânsito Brasileiro}

O Código de Trânsito Brasileiro(CTB) é um documento legal que foi instituído em 1997 pelo presidente da República por meio da Lei n 9.503 .Este documento define as responsabilidades das autoridades e órgãos ligados ao trânsito no Brasil.

O CTB é, basicamente, o conjunto de normas de trânsito do Brasil. Nele, encontramos todos os nossos deveres e direitos como cidadãos, desde a conduta adequada que devemos adotar no trânsito até os valores que serão cobrados por cada tipo de infração que, por ventura, viermos a cometer.

O CTB define(no capítulo 15 e no artigo 161) infração como: "a inobservância de qualquer preceito deste Código, da legislação complementar ou das resoluções do CONTRAN, sendo o infrator sujeito às penalidades e medidas administrativas indicadas em cada artigo, além das punições previstas no Capítulo XIX". 

\subsection{Tipos de Infração}

\subsection{Registro de Infrações de Trânsito}

    \subsubsection{Como é feito atualmente}

    \subsubsection{Registro Nacional de Infrações de Trânsito}

    \subsubsection{Estrutura de uma infração de trânsito}