\section{Blockchain}

    Antes de apresentar os conceitos relacionados diretamento ao Blockchain, precisamos primeiro elucidar o conceito de Sistema distribuído, pois este serve
    base para a compreensão do conceito e dos princípios existentes no blockchain.

\subsection[]{Sistemas Distribuídos}

    Varias definições de sistemas distribuidos foram dadas nos diversos materiais disponíveis na litetura.Para o propósito deste trabalho, pode-se dizer que: "Um sistema distribuído é uma coleção de 
    elementos computacionais autônomos que para os usuários parecem ser um único sistema coerente. Um elemento computacional, normalmente chamado de nó, pode
    ser tanto um dispositivo de hardware ou um processo de software.\cite{sistemas_distribuidos_tanembaum}

    Um nó pode ser definido como um "jogador individual",onde em um sistema distribuído todos os nós tem a capacidade de enviar e receber mensagem entre eles. Estes
    nós podem ser honestos, faltantes ou maliciosos e podem ter sua própria memória e processador. Um nó que apresenta um comportamento arbitrário também é
    chamado de Nó Bizantino.\cite{mastering_blockchain}

    O maior desafio em sistemas distribuidos é a coordenação entre esses nós e a tolerância a falha.Pois mesmo que um determinado nó apresente uma cada de conexão
    ou caso este esteja faltante o sistema deve tolerar estes problemas e continuar seu funcionamento de forma a alcançar o objetivo estabelecido. Estes sistemas 
    são considerados tão desafiadores que o teorema de CAP foi desenvolvido e provado, e o mesmo determina que um sistema distribuído não pode ter todas as propriedades
    de forma conjunta.

    \subsubsection{Teorema de CAP}

        Este teorema também é conhecido como \textit{Brewer's Theorem},foi introduzido inicialmente em 1998 por Eric Brewer; e provado como um teorema em 2002 por
        Seth Gilbert e Nancy Lynch.

        O teorema determina que um sistema distribuído não pode ter Consistência, Disponibilidade e Tolerância a Partição de forma simultânea \cite{brewer_distributed}.A
        seguir cada um destes é explicado de forma rápida:

        \begin{itemize}
            \item \textbf{Consistência:} essa propridade garante que todos os nós presentes em um sistema distribuido possuem uma cópia única da última informação;
            \item \textbf{Disponibilidade:} essa propriedade garante que um sistema está disponível, acessível para uso, e aceitando novas requisições e respondendo
        com os dados sem nenhuma falha e sempre quando é requisitado;
            \item \textbf{Tolerância a Partição:} garante que caso um grupo de nós falhe, o sistema distribuido ainda continua a operar corretamente.
        \end{itemize}
            

\subsection[]{Introdução Ao Blockchain}

    \subsubsection{Definição}

    \subsubsection{Elementos do Blockchain}

    \subsubsection{Tipos de Blockchain}

    \subsubsection{Transações}

    \subsubsection{Bloco}

        \subsubsection{A estrutura do bloco}
                % Estrutura do cabeçalho de um bloco

        \subsubsection{The genesis block}
    
    \subsubsection{Descentralização}

    
        \subsubsection{Descentralização no Blockchain}

        \subsubsection{Ecossistema da descentralização}

        \subsubsection{Smart Contract}

        \subsubsection{Organização descentralizada}

        \subsubsection{Aplicações descentralizadas}

        \subsubsection{Plataformas para descentralização}

\subsection{Teoria criptográfica}

    \subsubsection{Pilares}

    \subsubsection{Criptografia Simétrica}

    \subsubsection{Criptografia Assimetrica}
    
    \subsubsection{Funções de Hash}

    \subsubsection{Chaves Públicas e Privadas}

        \subsubsection{RSA}