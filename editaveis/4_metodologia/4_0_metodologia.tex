\chapter[Metodologia]{Metodologia}

\section{Metodologia de Desenvolvimento}

Para o desenvolvimento do trabalho será adotada uma metodologia híbrida, contendo diferentes ferramentas de abordagens de desenvolvimento de software. Essa personalização tem o objetivo de otimizar o desenvolvimento das tarefas no contexto em que esse trabalho é desenvolvido, tendo como uma das motivações principais para realizar essa personalização o fato do trabalho não ser construído em grupo.

Será utilizada a metodologia KanBan, com algumas personalizações, para gerenciar a execução de atividades. Os cartões utilizados neste quadro estarão representados no formato de \textit{User Story}, comumente utilizada na metodologia SCRUM, com seu conteúdo sendo composto por: uma descrição no formato da sentença \textbf{Eu como <quem>, quero <o que>, para <ação>}, as \textit{tasks} a serem desenvolvidas, o tempo limite para sua realização e a categoria da mesma.

O quadro será composto por 4 diferentes colunas: A fazer, Fazendo, Aguardando Validação e Finalizados. A imagem a seguir exemplifica o fluxo de desenvolvimento no quadro \textit{KanBan} que será utilizado no trabalho.

    \begin{figure}[H]
         \centering
         \includegraphics[scale=0.4]{figuras/capitulo_4/kanban_exemplo.png}
         \caption{KanBan representado na ferramenta \href{https://trello.com/}{Trello}}
         \label{fig:kanban_trello_exe}
    \end{figure}

A coluna \textbf{A Fazer} funcionará como um \textit{Backlog} das funcionalidades que serão desenvolvidas ao decorrer do trabalho (possíveis \textit{bugs} ou melhorias que surjam ao decorrer do projeto também serão adicionadas à essa coluna), onde os cartões posicionados acima são os considerados de maior prioridade para serem desenvolvidos. A coluna \textbf{Fazendo} será responsável por indicar o cartão que está que está em processo de desenvolvimento. Já a \textbf{Aguardando Validação} indicará o cartão que está passando por um processo de validação, como por exemplo a realização de testes com usuários reais para validar o funcionamento implementado em relação ao descrito no cartão. E a coluna \textbf{Finalizados} conterá o histórico das funcionalidades já implementadas e que foram validadas de acordo com os testes planejados.

Para melhor monitorar o andamento do trabalho, o planejamento dos cartões será feito por meio de iterações semanais. Neste planejamento semanal poderão ocorrer mudanças nas prioridades dos cartões presentes no quadro \textit{KanBan}, inserção de novos cartões na fila \textbf{A fazer}, remoção do quadro cartões que são considerados não necessários ou que precisem ser desmembrados em outros cartões pela complexidade da atividade. 

\section{Políticas para o desenvolvimento do Projeto}

Toda a aplicação terá sua construção feita na plataforma \href{https://github.com/}{GitHub}
em um repositório fechado. As issues do GitHub serão utilizadas como os cartões do KanBan para o cadastro das informações necessárias, e serão gerenciados de forma visual por meio da ferramenta \href{https://www.zenhub.com/}{ZenHub}.
 
A cada adição de funcionalidade ao sistema um novo \textit{commit} será feito para um melhor versionamento e controle do que será desenvolvido, e cada \textit{commit} terá seu texto no padrão \textbf{\#IdDaIssue Descrição}, como no seguinte caso:  “\#1 - Creating user structure”.

O repositório do projeto contará com duas branches principais para o desenvolvimento, sendo elas: \textit{master} e \textit{devel}. A branch \textit{master} conterá a versão estável do projeto, tendo seu conteúdo proveniente da branch de desenvolvimento (\textit{devel}), que será utilizada para o desenvolvimento do projeto, após a validação do que foi desenvolvido. Por decisões de projeto, nenhum \textit{commit} poderá ser feito diretamente na \textit{branch} \textit{master}, onde para inserir novos arquivos deve ser realizado um \textit{merge/pull request} tendo como origem das informações a \textit{branch devel}.
