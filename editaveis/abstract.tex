\begin{resumo}[Abstract]
 \begin{otherlanguage*}{english}
  
Blockchain became popular due to the widespread use of cryptocoins. The Bitcoin cryptocoin was introduced in mid-2009 and offered a real test field for blockchain because Bitcoin relies on it to digitally provide the same features of a regular currency. However, it should be noted that the blockchain technology has a potential of application in different context. This potential use of blockchain technology has stimulated various platforms and communities to support the development of these applications, allowing various applications to be created and to spread the technology in many areas. The records made in a blockchain are immutable, and these records are placed in blocks linked to previous records after the consensus of other nodes present in the network, which does not require the presence of a third party to authenticate the authenticity of the presented information. The present work proposes the implementation of a decentralized application (Dapp) for the registration of traffic infractions according to the norms present in the Brazilian Traffic Code, demonstrating the feasibility of the use of this new technology in a context of the daily life.
  

   \vspace{\onelineskip}
 
   \noindent 
   \textbf{Key-words}: Blockchain. Smart Contracts. Ethereum. Traffic Infractions.
 \end{otherlanguage*}
\end{resumo}
