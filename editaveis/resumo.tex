\begin{resumo}
O termo Blockchain se popularizou em função das negociações com criptomoedas, de forma mais notória a moeda digital Bitcoin que surgiu em meados de 2009, porém sua tecnologia possui um potencial de aplicação em diversos contextos. Este potencial de uso da tecnologia fez com que diversas plataformas e comunidades de apoio ao desenvolvimento destas aplicações surgissem, possibilitando que diversas aplicações fossem criadas e a tecnologia difundida cada vez mais em diversas áreas. Os registros feitos dentro de um blockchain são imutáveis, uma vez que estes registros são posicionados em blocos e vinculados a registros anteriores após o consenso de outros nós presentes na rede, o que torna não necessária a presença de uma terceira parte para autenticar a veracidade das informações apresentadas. O presente trabalho propõe a implementação de uma aplicação descentralizada (Dapp) para o registro de infrações de trânsito de acordo com as normas presentes no Código de Trânsito Brasileiro, demonstrando a viabilidade do uso desta nova tecnologia em um contexto presente no cotidiano das pessoas.
 \vspace{\onelineskip}
    
 \noindent
 \textbf{Palavras-chaves}: Blockchain. Contratos Inteligentes. Ethereum. Infrações de Trânsito.
\end{resumo}
